\section{Generation of Models}

Due to both confidentiality and technical reasons, it is difficult to obtain real-world industrial models and queries. Also, using confidential data sets hamstrings the reproducibility of the conducted benchmarks. Therefore, we used an artificial data set which mimics real-world models.

..

The generator capable of generating railroad instance models of different sizes. It is capable of generating models in different formats, including EMF, OWL, RDF and SQL. 

For Neo4j (\autoref{subsec:neo4j}) and Titan (\autoref{subsec:titan}), we expanded the generator with a module that can generate property graphs. The generator creates a graph in a Neo4j database and uses the Blueprints library's \texttt{GraphMLWriter} and \texttt{GraphSONWriter} classes to serialize the graph to \graphml{} and Blueprints \graphson{} formats. It is also capable of serializing the graph to Faunus \graphson{} format.


- based on the \tb{}, I created an extended version for distributed systems. 
  - cite ASE article
  - TrainBenchmark
  - models in different format
  - different technologies
  - different workflow (queries, model manipulation)
  - measure eponse time, not throughput
  - we increate the model's size --> query results size increases
  - read check edit check phases
  - the benchmark runs on a coordinator and delegates the operations to a distributed system
  - one user uses the entire cluster (for benchmarking purposes)
