\appendix
 
\chapter{Graph Formats}

In this chapter, we provide examples for the different graph serialization formats, including property graphs and RDF graphs. The examples describe a small instance model based on the railway system metamodel, shown on \figref{trainbenchmark-instancemodel-subgraph}.

\picSmall{trainbenchmark-instancemodel-subgraph}{An example graph based on the railway system metamodel}
 
\section{Property Graph Formats}
\label{property-graph-formats}

\subsection{\graphml{}}
\label{graphml-example}

The \graphml{} format \cite{GraphML} is the most widely used graph representation format, based on XML (Extensible Markup Language). It has strong tooling support between graph databases and graph visualizing tools.

\lstset{language=XML,breaklines=true}
\begin{lstlisting}[caption=A graph based on the railway system metamodel stored in \graphml{} format]
<?xml version="1.0" encoding="UTF-8"?>
<graphml xmlns="http://graphml.graphdrawing.org/xmlns" xmlns:xsi="http://www.w3.org/2001/XMLSchema-instance" xsi:schemaLocation="http://graphml.graphdrawing.org/xmlns http://graphml.graphdrawing.org/xmlns/1.1/graphml.xsd">
  <key id="type" for="node" attr.name="type" attr.type="string" />
  <graph id="G" edgedefault="directed">
    <node id="1">
      <data key="type">Sensor</data>
    </node>
    <node id="2">
      <data key="type">Route</data>
    </node>
    <node id="3">
      <data key="type">SwitchPosition</data>
    </node>
    <node id="4">
      <data key="type">Switch</data>
    </node>
    <edge id="0" source="2" target="1" label="ROUTE_ROUTEDEFINITION" />
    <edge id="1" source="2" target="3" label="ROUTE_SWITCHPOSITION" />
    <edge id="2" source="3" target="4" label="SWITCHPOSITION_SWITCH" />
    <edge id="3" source="4" target="1" label="TRACKELEMENT_SENSOR" />
  </graph>
</graphml>
\end{lstlisting}

\subsection{Blueprints \graphson{}}
\label{blueprints-graphson-example}

Blueprints \graphson{} \cite{BlueprintsGraphSON} is a JSON-based (JavaScript Object Notation) format. It is not as well supported as the GraphML format (\autoref{graphml-example}), but it is less verbose and more readable.

\lstset{language=json,firstnumber=1}
\begin{lstlisting}[caption=A graph based on the railway system metamodel stored in Blueprints \graphson{} format]
{
  "vertices":[
    {
      "type":"Sensor",
      "_id":1,
      "_type":"vertex"
    },
    {
      "type":"Route",
      "_id":2,
      "_type":"vertex"
    },
    {
      "type":"SwitchPosition",
      "_id":3,
      "_type":"vertex"
    },
    {
      "type":"Switch",
      "_id":4,
      "_type":"vertex"
    }
  ],
  "edges":[
    {
      "_id":0,
      "_type":"edge",
      "_outV":2,
      "_inV":1,
      "_label":"ROUTE_ROUTEDEFINITION"
    },
    {
      "_id":1,
      "_type":"edge",
      "_outV":2,
      "_inV":3,
      "_label":"ROUTE_SWITCHPOSITION"
    },
    {
      "_id":2,
      "_type":"edge",
      "_outV":3,
      "_inV":4,
      "_label":"SWITCHPOSITION_SWITCH"
    },
    {
      "_id":3,
      "_type":"edge",
      "_outV":4,
      "_inV":1,
      "_label":"TRACKELEMENT_SENSOR"
    }
  ]
}
\end{lstlisting}

\subsection{Faunus \graphson{}}
\label{faunus-graphson-example}

In the Faunus \graphson{} format \cite{FaunusGraphSON}, each line is a separate JSON (JavaScript Object Notation) document representing a vertex in the graph. This way, the file can be splitted to blocks efficiently and processed on Hadoop nodes in a parallel fashion.   

\begin{lstlisting}[caption=A graph based on the railway system metamodel stored in Faunus \graphson{} format]
{"type":"Sensor","_id":1,"_outE":[],"_inE":[{"_id":0,"_outV":2,"_label":"ROUTE_ROUTEDEFINITION"},{"_id":3,"_outV":4,"_label":"TRACKELEMENT_SENSOR"}]}
{"type":"Route","_id":2,"_outE":[{"_id":0,"_inV":1,"_label":"ROUTE_ROUTEDEFINITION"},{"_id":1,"_inV":3,"_label":"ROUTE_SWITCHPOSITION"}],"_inE":[]}
{"type":"SwitchPosition","_id":3,"_outE":[{"_id":2,"_inV":4,"_label":"SWITCHPOSITION_SWITCH"}],"_inE":[{"_id":1,"_outV":2,"_label":"ROUTE_SWITCHPOSITION"}]}
{"type":"Switch","_id":4,"_outE":[{"_id":3,"_inV":1,"_label":"TRACKELEMENT_SENSOR"}],"_inE":[{"_id":2,"_outV":3,"_label":"SWITCHPOSITION_SWITCH"}]}
\end{lstlisting}

\section{Semantic Graph Formats}

\subsection{RDF/XML}
\label{rdfxml-example}

RDF/XML is an XML-based (Extensible Markup Language) format for serializing RDF triples.

\lstset{language=XML,breaklines=true}
\begin{lstlisting}[caption=A graph based on the railway system metamodel stored in RDF format]
<?xml version="1.0" encoding="UTF-8"?>
<rdf:RDF
	xmlns="http://www.semanticweb.org/ontologies/2011/1/TrainRequirementOntology.owl#"
	xmlns:rdfs="http://www.w3.org/2000/01/rdf-schema#"
	xmlns:swrl="http://www.w3.org/2003/11/swrl#"
	xmlns:swrlb="http://www.w3.org/2003/11/swrlb#"
	xmlns:xsd="http://www.w3.org/2001/XMLSchema#"
	xmlns:owl="http://www.w3.org/2002/07/owl#"
	xmlns:rdf="http://www.w3.org/1999/02/22-rdf-syntax-ns#">

<rdf:Description rdf:about="http://www.semanticweb.org/ontologies/2011/1/TrainRequirementOntology.owl">
	<rdf:type rdf:resource="http://www.w3.org/2002/07/owl#Ontology"/>
</rdf:Description>

<rdf:Description rdf:about="http://www.semanticweb.org/ontologies/2011/1/TrainRequirementOntology.owl#Segment">
	<rdf:type rdf:resource="http://www.w3.org/2002/07/owl#Class"/>
	<rdfs:subClassOf rdf:resource="http://www.semanticweb.org/ontologies/2011/1/TrainRequirementOntology.owl#Trackelement"/>
</rdf:Description>

<rdf:Description rdf:about="http://www.semanticweb.org/ontologies/2011/1/TrainRequirementOntology.owl#Switch">
	<rdf:type rdf:resource="http://www.w3.org/2002/07/owl#Class"/>
	<rdfs:subClassOf rdf:resource="http://www.semanticweb.org/ontologies/2011/1/TrainRequirementOntology.owl#Trackelement"/>
</rdf:Description>

<rdf:Description rdf:about="http://www.semanticweb.org/ontologies/2011/1/TrainRequirementOntology.owl#1">
	<rdf:type rdf:resource="http://www.semanticweb.org/ontologies/2011/1/TrainRequirementOntology.owl#Sensor"/>
</rdf:Description>

<rdf:Description rdf:about="http://www.semanticweb.org/ontologies/2011/1/TrainRequirementOntology.owl#2">
	<rdf:type rdf:resource="http://www.semanticweb.org/ontologies/2011/1/TrainRequirementOntology.owl#Route"/>
</rdf:Description>

<rdf:Description rdf:about="http://www.semanticweb.org/ontologies/2011/1/TrainRequirementOntology.owl#3">
	<rdf:type rdf:resource="http://www.semanticweb.org/ontologies/2011/1/TrainRequirementOntology.owl#Switch"/>
</rdf:Description>

<rdf:Description rdf:about="http://www.semanticweb.org/ontologies/2011/1/TrainRequirementOntology.owl#4">
	<rdf:type rdf:resource="http://www.semanticweb.org/ontologies/2011/1/TrainRequirementOntology.owl#SwitchPosition"/>
</rdf:Description>

<rdf:Description rdf:about="http://www.semanticweb.org/ontologies/2011/1/TrainRequirementOntology.owl#3">
	<TrackElement_sensor rdf:resource="http://www.semanticweb.org/ontologies/2011/1/TrainRequirementOntology.owl#1"/>
</rdf:Description>

<rdf:Description rdf:about="http://www.semanticweb.org/ontologies/2011/1/TrainRequirementOntology.owl#4">
	<SwitchPosition_switch rdf:resource="http://www.semanticweb.org/ontologies/2011/1/TrainRequirementOntology.owl#3"/>
</rdf:Description>

<rdf:Description rdf:about="http://www.semanticweb.org/ontologies/2011/1/TrainRequirementOntology.owl#2">
	<Route_routeDefinition rdf:resource="http://www.semanticweb.org/ontologies/2011/1/TrainRequirementOntology.owl#1"/>
	<Route_switchPosition rdf:resource="http://www.semanticweb.org/ontologies/2011/1/TrainRequirementOntology.owl#4"/>
</rdf:Description>

</rdf:RDF>
\end{lstlisting}

\section{Mapping Ecore to Property Graphs}
\label{trainbenchmark-mapping}

Mapping the Ecore kernel's concepts to property graphs is not a trivial task. We developed the property graph generator module for the \tb{} based on the railroad system's Ecore metamodel (\autoref{railroad-system}), which meant the Ecore concepts had to be mapped to property graphs. Following the mapping defined in \autoref{ecore-mapping}, we created the equivalent instance models for property graphs as well. Below, we provide some examples about the mapping:
\begin{itemize}
  \item \verb+Segment+ is an \verb+EClass+ instance. In a property graph, types cannot be represented explicitly. Instead, for each node representing a \verb+Segment+ instance, we add a \verb+type+ property with the value \verb+Segment+.
  \item \verb+Segment_length+ is an \verb+EAttribute+ instance. For each graph node representing a \verb+Segment+, we define a property with the value \verb+Segment_length+.
  \item \verb+TrackElement_Sensor+ is an \verb+EReference+ instance. For each edge representing a \verb+TrackElement_Sensor+ instance, we add the \verb+TRACKELEMENT_SENSOR+ label.
  \item \verb+EInt+ in an \verb+EDataType+ instance. Each attribute with this type, e.g.\ the \verb+Sensor+ class' \verb+Segment_length+ attribute, is defined with the Java primitive type \verb+int+.
\end{itemize}
