\section{Tooling}
\label{tooling}   

To be able to focus on the distributed aspects, of the system, we aimed to build \iqd{} on top of \eiq{}'s pattern language (IQPL) and its Rete network generator. Also, \eiq{} has an Eclipse-based user interface for defining and executing queries.

For \iqd{}, we plan to provide the same tooling environment. Also, for the allocation of Rete nodes, we created an Eclipse-based editor and viewer.



% see also http://components.neo4j.org/neo4j/snapshot/apidocs/org/neo4j/graphdb/PropertyContainer.html



% TODO move to TB

\subsection{Runtime Model-based Dashboard}
\label{dashboard}

To aid the system's dynamic capabilities, we plan to develop a runtime model-based dashboard to monitor the state of \iqd{}'s nodes. Currently, the \iqd{} tooling generates an architecture file (\texttt{arch}), which is used for deploying the distributed pattern matcher.

This file contains the Rete network's layout and its allocation in the cloud (as of now, the latter is defined manually). \iqd{} uses the architecture description for instantiating the Rete network and initializing the middleware (\figref{incqueryd-architecture-dashboard}).

\picTiny{incqueryd-architecture-dashboard}{Architecture of \iqd{} with a runtime dashboard}

To provide live feedback, we will adopt a \emph{live} architecture model. The live model will provide real-time details about the systems' current state, including the local resources on each server, the Rete nodes' memory consumption and so on.
