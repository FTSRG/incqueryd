\chapter{Related Work}
\label{chap:related-work}

A wide range of special languages have been developed to support \emph{graph-based} representation and querying of computer data. This chapter collects the research and development works that are related to \iqd{}.

\section{Eclipse-based Tools}

A class-diagram like modeling language is Ecore of the EMF (Eclipse Modeling Framework, discussed in \autoref{emf}), where classes, references between them and attributes of classes describe the domain. 
Extensive tooling helps the creation and transformation of such domain models. For EMF models, OCL (Object Constraint Language) is a declarative constraint description and query language that can be evaluated with the local-search based Eclipse OCL~\cite{EclipseOCL} engine. To address scalability issues, \emph{incremental} impact analysis tools~\cite{OCLIA} have been developed as extensions or alternatives to Eclipse OCL.

% \section{Semantic Web and NoSQL}

% Outside the Eclipse ecosystem, the Resource Description Framework (RDF~\cite{website:rdf_standard}) is developed to support the description of instances of the semantic web, assuming sparse, ever-growing, incomplete data. Semantic models are built up from triple statements and they can be queried using the SPARQL~\cite{SPARQL} graph pattern language with tools like Sesame~\cite{sesame} or Virtuoso~\cite{openvirtuoso}. 
% 
% Property graphs~\cite{DBLP:journals/corr/abs-1006-2361} provide a more general way to describe graphs by annotating vertices and edges with key-value properties. 
% 
% Such data structures can be stored in graph databases like Neo4j~\cite{neo4j} which provides the Cypher~\cite{cypher} query language. Even though big data storage (usually based on MapReduce) provides fast object persistence and retrieval, query engines realized directly on these data structures do not provide dedicated support for incremental query evaluation. 

\section{Rete Implementations}

%In the context of event-based systems, distributed evaluation engines were proposed earlier~\cite{message-passing-rete}. However they scaled up in the number of rules~\cite{mapreduce-rete} rather than in the number of data elements. 

As a very recent development, Rete-based caching approaches have been proposed for the processing of Linked Data (bearing the closest similarity of our approach). \mbox{INSTANS}~\cite{INSTANS2012} uses this algorithm to perform complex event processing (formulated in SPARQL) on RDF data, gathered from distributed sensors.

Diamond~\cite{miranker2012diamond} uses a \emph{distributed Rete network} to evaluate SPARQL queries on Linked Data, but it lacks an indexing middleware layer so their main challenge is efficient data traversal.

The conceptual foundations of our approach as based on \eiq{}~\cite{models10}, a tool that evaluates graph patterns over EMF models using Rete. Up to our best knowledge, \iqd{} is the first approach to promote distributed scalability by \emph{distributed incremental query evaluation} in the context of model-driven engineering. As the architecture of \iqd{} separates the data store from the query engine, we believe that the scalable processing of RDF and property graphs can open up interesting applications outside of the MDE world.

Acharya et al.\ described a Rete network mapping for fine-grained and medium-grained message-passing computers~\cite{message-passing-rete}. The medium-grained computer connected processors in a crossbar architecture, while our approach use computers connected by gigabit Ethernet. The paper published benchmark results of the medium-grained solution, but these are based only on simulations.

\section{Benchmarking}

In the research work undertaken in the Budapest University of Technology and Economics, numerous benchmarks were designed and elaborated for graph pattern matching and graph transformation~\cite{VSV05b, STTT10}.

The distributed \tb{} used in this report builds on the most recent results, published in~\cite{ASE2013}.

