\chapter{Conclusions}
\label{chap:conclusions}

This chapter summarizes the contributions presented in the report. 

\section{Summary of Contributions}

We presented \iqd{}, a novel approach to adapt distributed incremental query techniques to large and complex model driven software engineering scenarios. Our proposal is based on a distributed Rete network that is decoupled from sharded graph databases by a middleware layer, and its feasibility has been evaluated using a benchmarking scenario of on-the-fly well-formedness validation of software design models. The results are promising as they show nearly instantaneous query re-evaluation as model sizes grow well beyond $10^7$ elements.

% TODO separate

During the research and development of \iqd{}'s prototype, I achieved the following results.

\subsection{Scientific Contributions}

I achieved the following scientific contributions:

\begin{itemize}
  \item I designed and implemented a \emph{distributed, asynchronous version of the Rete algorithm}. 
  \item I proposed a novel architecture for building a distributed, scalable, incremenetal graph query engine over different storage backends. % incremental query as a service
  \item I conducted a benchmark to measure \iqd{}'s \emph{response time and scalability characteristics}. For the benchmark's baseline, I created \emph{distributed non-incremental benchmark scenarios}. 
  \item I extended the termination protocol used \eiq{} to work in a distributed environment. 
  \item I extended the \tb{} to work in a distributed environment. 
\end{itemize}

\subsection{Practical Accomplishments}

I achieved the following practical accomplishements:

\begin{itemize}
  \item Based on the Rete algorithm, I created a \emph{distributed incremental query engine's prototype}, which is not only detached from the data storage backend, but also agnostic to the storage backend's data model. To prove this, the query engine was tested with both property graphs and RDF graphs.
  \item I extended the \tb{} with a \emph{new instance model generator}, which can produce property graphs and serialize them in various formats: \graphml{}, Blueprints \graphson{} and Faunus \graphson{}. % (\autoref{sec:property-graph-formats}).
  \item I implemented \iqd{}'s prototype, including the storage, the middleware and the query layer. I wrote more than 3000 lines of Java code and approximately 500 lines of configuration and deployment scripts.
%  \item Based on \eiq{} and István Ráth's work, I created an \emph{Eclipse-based environment} to provide automated deployment of the Rete network. This allows the user to define complex queries in IQPL, a high-level pattern language. % TODO SzG define own work
  \item I elaborated automated deployment technologies based on \eiq{}'s existing technologies.
  \item I experimented with modern non-relational database management systems with a focus on NoSQL graph databases and triple stores. For the purpose of benchmarking different tools, I implemented scripts to install various graph storages. 
  \begin{itemize}
    \item I implemented scripts to install the \emph{Titan graph database and its ecosystem} on a cluster. Titan's ecosystem includes technologies on different maturity levels, including the Apache Cassandra database, the Apache Hadoop MapReduce framework with the HDFS distributed file system, the TinkerPop graph framework and the Faunus graph analytics engine. I formulated the necessary the Gremlin queries and created the connector class in \iqd{}'s middleware.
    \item I implemented scripts to install the \emph{4store triplestore} on a cluster. I formulated the necessary the SPARQL queries and created the connector class in \iqd{}'s middleware. 
    \item I deployed a manually sharded \emph{Neo4j cluster}. I formulated the appropriate Cypher queries and created the connector class in \iqd{}'s middleware to access Neo4j.
  \end{itemize}
  \item I implemented scripts for \emph{automating the benchmark} and \emph{operating a cluster of Akka microkernels}.
\end{itemize}

\section{Limitations}

\iqd{}'s current implementation has some limitations, the most important being the following:

\begin{itemize}
  \item The Rete nodes are allocated manually.
  \item Only a subset of the nodes defined in the Rete algorithm are implemented. %No \emph{check} expressions. 
  \item The Eclipse-based tooling does not cover the whole workflow. The user is required to do some manual work, e.g.\ running scripts manually.
\end{itemize}

\section{Future work}
\label{future-work}

For future work, we plan on providing more sophisticated automation for sharded Ecore models, and further exploring advanced optimization challenges such as dynamic reconfiguration and fault tolerance.

We also plan experiment with programming languages that are better suited to asynchronous algorithms (e.g.\ Erlang and Scala) and try different database systems (e.g. MongoDB) as our storage layer.

We plan to extend the distributed \tb{} to model different real-world workloads, e.g.\ simulating multiple users issuing concurrent requests.

\begin{itemize}
  \item Use programming languages better suited for asynchronous processing, e.g.\ Scala, a Java-based functional object-oriented programming language.
  \item Implement \emph{node sharding}.
  \item Automatic allocation of Rete nodes using CSP (Constraint Satisfaction Problem), DSE (Design Space Exploration) \cite{DSE11}
  \item Implement \emph{live monitoring}. 
\end{itemize}
