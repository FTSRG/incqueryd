\chapter{Conclusions}
\label{chap:conclusions}

\section{Summary of contributions}

\subsection{Own work}

\section{Limitations}

\section{Future directions}

We presented \iqd{}, a novel approach to adapt distributed incremental query techniques to large and complex model driven software engineering scenarios. Our proposal is based on a distributed Rete network that is decoupled from sharded graph databases by a middleware layer, and its feasibility has been evaluated using a benchmarking scenario of on-the-fly well-formedness validation of software design models. The results are promising as they show nearly instantaneous query re-evaluation as model sizes grow well beyond $10^7$ elements.
For future work, we plan on providing more sophisticated automation for sharded Ecore models, and further exploring advanced optimization challenges such as dynamic reconfiguration and fault tolerance.
We also plan experiment with programming languages that are better suited to asynchronous algorithms (e.g.\ Erlang and Scala) and try different database systems (e.g. 10gen MongoDB) as our storage layer.

