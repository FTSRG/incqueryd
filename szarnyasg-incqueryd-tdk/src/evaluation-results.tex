\section{Results}
\label{evaluation-results}

In the following section, we discuss the results of our benchmark.

\subsection{Result visualizations}

The $x$ axis shows the models size (number of model elements), the $y$ axis shows execution time. Both axes are logarithmic. 

%As expected, the $\mathit{Read}$ phase take about the same time for both scenarios, and \iqd{} is about half an order of magnitude slower when evaluating the query at first ($\mathit{Check}_0$ phase) due to the Rete construction overhead. However, \iqd{} is several orders of magnitude faster during the $\mathit{Edit}_i-\mathit{Check}_i$ cycles, making on-the-fly query (re)evaluation feasible even for models larger than 50 million elements. Once initialized, \iqd{} scales linearly, since query response times for growing models can be kept low by adding additional computers for hosting Rete nodes.

%4.5.1 prezentálni minimális magyarázattal (ami callout lenne), log scale, OOM differences


%\pic{benchmark/Read_RouteSensor}{Execution times for the $\mathit{Read}$ phase}
%\figref{benchmark/Read_RouteSensor}

%\pic{benchmark/Check0_RouteSensor}{$\mathit{Check}_0$ phase}
%\figref{benchmark/Check0_RouteSensor}



\pic{benchmark/BatchValid_RouteSensor}{Execution time for load and first validation}

\figref{benchmark/BatchValid_RouteSensor}



\pic{benchmark/OnTheFly_RouteSensor}{Execution time for transformation and revalidation} %($\mathit{Edit}$ and $\mathit{Check}_1$  phase)}

\figref{benchmark/OnTheFly_RouteSensor}



\pic{benchmark/OnTheFlyEdit_RouteSensor}{Execution time for transformation} %On-the-fly revalidation, $\mathit{Edit}$ phase}

\figref{benchmark/OnTheFlyEdit_RouteSensor}


 
\pic{benchmark/OnTheFlyCheck_RouteSensor}{Execution time of the revalidation} % ($\mathit{Check}_1$ phase)}

\figref{benchmark/OnTheFlyCheck_RouteSensor}



\pic{benchmark/BatchTrafo_RouteSensor}{Total execution time for 50 validations}

\figref{benchmark/BatchTrafo_RouteSensor}

 



\section{Result Analysis}



- összekapcsolni a látott eredményeket a rendszer felépítésével
- lapos karakterisztika azt mutatja, hogy működik az elv

The results prove that while network latency is present, the distributed Rete network still allows quick on-the-fly model validation operations. 

\subsection{Memory Consumption}

During our experiments, we measured cases where the Java Virtual Machine ran out of memory or had just enough memory, resulting in \texttt{OutOfMemoryError: Java heap space} and \texttt{OutOfMemoryError: GC overhead limit exceeded} exceptions, respectively. Introducing fault-tolerance mechanism for these cases is subject to future work (\autoref{future-work}).  
  
For incremental tools, the execution time is approximately proportional to \emph{the number of affected model elements}. For non-incremental tools, it is proportional to the \emph{size of the model}. 



* inkrementális esetben a APPROX MÓDOSÍTÁSSAL arányos a lekérdezés ideje.
* nem-inkrementális esetben a MODELLEL arányos
* Rete akkor jó, ha kicsiket módosít a user


% \subsection{Results Using Neo4j}
% 
% \pic{benchmark/neo/BatchTrafo_RouteSensor}{Batch transformation}
% 
% \figref{}
% 
% \pic{benchmark/neo/OnTheFly_RouteSensor}{On-the-fly revalidation ($\mathit{Edit}$ and $\mathit{Check}_1$  phase)}
% 
% \figref{}

\subsection{Threats to validity}
\label{threats-to-validity}

4.5.3 threats to validity
  - befolyásoló tényezők
    - caching off
    - keresztbe terhelés, tranziens hibák 5x mérés, ...
