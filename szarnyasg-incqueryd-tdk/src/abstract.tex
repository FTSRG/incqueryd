%----------------------------------------------------------------------------
% Abstract in Hungarian
%----------------------------------------------------------------------------
\chapter*{Kivonat}\addcontentsline{toc}{chapter}{Abstract in Hungarian}

\cite{Izso:2013:IIG:2487766.2487772}
\cite{4store}

Az adatintenzív alkalmazások nagy kihívása a lekérdezések hatékony kiértékelése. A modellvezérelt szoftvertervezés (MDE) során az eszközök és a transzformációk különböző bonyolultságú lekérdezéssekkel dolgoznak. Míg a szoftvermodellek mérete és komplexitása folyamatosan nő, a hagyományos MDE eszközök gyakran nem skálázódnak megfelelően, így csökkentve a fejlesztés produktivitását és növelve a költségeket.

Ugyan az újgenerációs, ún. NoSQL adatbázis-kezelő rendszerek többsége képes horizontális skálázhatóságra, az ad-hoc lekérdezéseket nem támogatja olyan hatékonyan, mint a relációs adatbázisok. Mivel a modellvezérelt alkalmazások tipikusan komplex lekérdezéseket futtatnak, a NoSQL adatbázis-kezelők közvetlenül nem használhatók ilyen célra.

Diplomatervem célja, hogy az \incquery{}-ben alkalmazott inkrementális gráfmintaillesztő algoritmust elosztott, felhőalapú infrastruktúrára implementáljam. Az \incqueryD{} prototípus skálázható, így képes több számítógépből álló fürtön nagy modelleket kezelni és komplex lekérdezések hatékonyan kiértékelni. Az elképzelés életképességét előzetes mérési eredményeink igazolják.

\vfill

%----------------------------------------------------------------------------
% Abstract in English
%----------------------------------------------------------------------------
\chapter*{Abstract}\addcontentsline{toc}{chapter}{Abstract in English}

Queries are the foundations of data intensive applications. In model-driven software engineering (MDE), model queries are core technologies of tools and transformations. As software models are rapidly increasing in size and complexity, traditional MDE tools frequently exhibit scalability issues that decrease productivity and increase costs.

While such scalability challenges are a constantly hot topic in the database community and recent efforts of the NoSQL movement have partially addressed many shortcomings, this happened at the cost of sacrificing the powerful ad-hoc query capabilities of SQL. Unfortunately, this is a critical problem for MDE applications, as their queries can be significantly more complex than in general database applications.

In my thesis work, I aim to address this challenge by adapting incremental graph search techniques -- known from the \incquery{} framework -- to the distributed cloud infrastructure. \incqueryD, my prototype system can scale up from a single-node tool to a cluster of nodes that can handle very large models and complex queries efficiently. The feasibility of my approach is supported by early experimental results.

\vfill

