\section{Deployment, configuration}
\label{sec:deployment-configuration}

\subsection{Tooling}

reuse of \eiq's components

Eclipse-based tree editor and yworks viewer

\myFigure{recipe}{The workflow of \eiq{} (blue) and \iqd{} (green)}

\subsection{Degrees of freedom}

database sharding, allocation of rete nodes -- orthogonal

choosing different dbs

\subsubsection{Different database implementations}

diffent storage backends are supported



\subsection{Workflow} % ''Problem pieces''

In the following part, we will describe the workflow behind the pattern matching process. Starting from a metamodel, an instance model and a graph pattern, we will describe the problem pieces that need to be solved for setting up an incremental, distributed pattern matcher.


\subsubsection{Analyze the model and the query}

\paragraph{Task.} IQPL (\iq{} Pattern Language)

IQPL -> EMF model

\paragraph{Implementation.} IQPL, EMF model, P-System 


\subsubsection{Build a Rete layout}

EMF model -> PSystem -> recipe -> rete layout

\paragraph{Task.}  

\paragraph{Implementation.} \eiq{}'s implementation 


\subsubsection{Allocate the Rete network in the cloud's nodes}

distribution, latency, throughput, .. 

\paragraph{Task.} 

\paragraph{Implementation.} CSP, DSE~\cite{DSE11}


\subsubsection{Bootstrap the system}

deploying actors, initiating the Rete net's processing workflow

\paragraph{Task.} 

\paragraph{Implementation.} Automatic bootstrapping and operation 



\myFigure{incqueryd-tooling-tree-editor}{The editor in \iqd's tooling}

\myFigure{incqueryd-tooling-yfiles-viewer}{The viewer in \iqd's tooling}
