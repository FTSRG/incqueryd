\section{Elaboration of the example}
\label{sec:elaboration}

We use the \textit{RouteSensor} query as our example. The query is shown as a graph pattern definition on \lstref{routesensor-iqpl} and visualized on \figref{routesensor-pattern}. Queries like this are typical in MDE applications (such as well-formedness validation or complex model transformations).

\subsection{Workflow}

Following the workflow defined in \autoref{subsec:workflow}, we will cover the actual steps for deploying and operating a distributed pattern matcher for the \textit{RouteSensor} query.

\subsubsection{Analyze the metamodel and the query}

The metamodel is shown on \figref{trainbenchmark-metamodel}. Using \eiq{}'s tooling, the textual representation (\texttt{routeSensor.arch}, see \lstref{routesensor-iqpl}) is analyzed and parsed to an EMF model (\figref{eiq-model}).

\myFigureSmall{eiq-model}{The EMF model generated from the pattern}

\subsubsection{Build a Rete layout}

Based on the query's EMF model, \eiq{}'s tooling builds PSystem and creates a Rete layout, that quarantees the satisfaction of the constraints. The Rete layout is shown on \figref{rete-routesensor-example-rete}. 

\subsubsection{Allocate the Rete network in the cloud's nodes} 

In \iqd{}'s current implementation, the Rete recipe's nodes are allocated manually on the cloud servers (called \textit{Machine}s). The allocation is currently defined in an architecture file (e.g.\ \texttt{routeSensor.arch}). The Rete nodes are associated with the machines with \textit{infrastructure mapping} edges.

\iqd{}'s tooling currently provides an Eclipse-based tree editor to define machines and the infrastructure mapping edges (\figref{incqueryd-tooling-tree-editor}).

\myFigureSmall{incqueryd-tooling-tree-editor}{The tree editor in \iqd's tooling}

The tooling is capable of visualizing the Rete network and its mapping to the machines (see \figref{incqueryd-tooling-yfiles-viewer})

\myFigureSmall{incqueryd-tooling-yfiles-viewer}{The yFiles viewer in \iqd's tooling}

\subsubsection{Bootstrap the system}

\iqd{}'s current implementation, the distributed system is initiated with a Bash script which launches the Akka microkernel on the appropriate nodes. The Akka actors representing the Rete network's nodes are deployed automatically by the \iqd{} \textit{Coordinator} node.
