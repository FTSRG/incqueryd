\section{Architecture of incremental queries}
\label{sec:architecture}

In the following, we will cover the challenges for building an incremental pattern matcher and present the architecture of \iqd{}.

%%%%%%%%%%%%%%%%%%%%%%%%%%%%%%%%%%%%%%%%%%%%%%%%%%%%%%%%%%%%%%%%%%%%%%%%%%%%%%%%%%%%%%%%%%%%%%%%%%%%


\subsection{Incremental query evaluation}
\label{rete}

Some queries, e.g.\ well-formedness constraints in MDE are evaluated many times, while the data sets they are evaluated on only changes to a small degree. In these cases, the idea of incremental query evaluation arises naturally: to speed up queries, we should not start the evaluation all over again. Instead, we should rely on the (partial) results derived during the previous executions of the query and process the changes that occured.
 
In practice, incremental query evaluation algorithms typically use data structures for caching the interim results. This  means that they usually consume more memory, in other words, they trade memory consumption for execution speed. This approach, called \emph{space--time tradeoff}, is well-known and widely used in computer science.

In the following, we provide an overview of the Rete algorithm, which forms the theoretical basis of \eiq{} and \iqd{}.

\subsubsection{Rete in general}

\iqd{} is based on the Rete algorithm, which provides a propagation network for incremental graph pattern matching\footnote{\emph{Rete} is Latin for \emph{net} or \emph{comb}.}. The algorithm was originally created by Charles Forgy~\cite{Forgy} for rule-based expert systems. Gábor Bergmann adapted the algorithm for EMF models and added many tweaks and improvements to it~\cite{BergmannRete}.

\myFigure{rete}{The structure of the Rete propagation network}

The Rete algorithm defines an asynchronous network of communicating nodes. This is essentially a dataflow network, with two types of nodes. Change notification objects (\emph{tokens}) are propagated to intermediate \emph{worker nodes} that perform operations, like filtering tokens based on constant expressions and performing join or antijoin operations based on their contents. The worker nodes store partial (interim) query results in their own memory. In contrast, \emph{production nodes} are terminators that provide an interface for fetching query results and also their changes (\emph{deltas}) \figref{rete}. %Connections between nodes can be \emph{local} (within one host) or \emph{remote} (when two Rete nodes are allocated to different hosts).

\subsubsection{Similar algorithms}

Along the original Rete algorithm, many algorithms were developed for incremental pattern matching. Rete itself has improved versions (Rete II, Rete III, Rete-NT), however, unlike the original algorithm, these are not publicly available. 

Other algorithms include TREAT \cite{Miranker:1991:OPT:627280.627434}, which aims at minimizing memory usage by using only indexers and dropping partial results, while having the same algorithmic complexity as Rete. Another candidate is the LEAPS \cite{Batory:1994:LA:899216} algorithm, which is claimed to provide better space--time complexity. However, we found that even LEAPS is difficult to understand and implement even on a single workstation, not to mention the distributed case. 

Because the Rete algorithm is well-known and well-understood by the \eiq{} team, we decided to build \iqd{} on the same foundation. Experimenting with improved versions or alternative approaches is subject to future work.

\subsubsection{Termination protocol}

As the Rete algorithm's change propagation is asynchronous, the system must also implement a \emph{termination protocol} to ensure that the query results can be retrieved consistently with the model state after a given transaction (i.e.\ by signaling when the update propagation has been terminated). \iqd{}'s current termination protocol works by adding a stack to the message. The stack registers each Rete node the message passes through. After the message reaches the production node, the termination protocol starts. Based on the content of the stack, acknowledgement messages are propagated back on the network. When all relevant indexer nodes (where the original notification token(s) started from) receive the acknowledge messages, the termination protocol finishes.

%%%%%%%%%%%%%%%%%%%%%%%%%%%%%%%%%%%%%%%%%%%%%%%%%%%%%%%%%%%%%%%%%%%%%%%%%%%%%%%%%%%%%%%%%%%%%%%%%%%%



\subsection{Notification mechanisms}
\label{notifications}

\emph{Model change notifications} are required by incremental query evaluation, thus model changes are captured and their effectes propagated in the form of \emph{notification objects} (NOs). The notifications generate \emph{tokens} that keep the Rete network's state consistent with the model. \iqd{}'s middleware layer facilitates notifications by providing a facade for model manipulation operations (\figref{incqueryd-architecture} \textcircled{3}).


